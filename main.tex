%----------------------------------------------------------------------------------------
%	PACKAGES AND OTHER DOCUMENT CONFIGURATIONS
%----------------------------------------------------------------------------------------

\documentclass[11pt]{scrartcl} % Font size

%----------------------------------------------------------------------------------------
%	PACKAGES AND OTHER DOCUMENT CONFIGURATIONS
%----------------------------------------------------------------------------------------

\usepackage{amsmath, amsfonts, amsthm} % Math packages

\usepackage{listings} % Code listings, with syntax highlighting

\usepackage[english]{babel} % English language hyphenation


\usepackage{booktabs} % Required for better horizontal rules in tables

\setlength\parindent{0pt} % Removes all indentation from paragraphs

\usepackage{enumitem} % Required for list customisation

\usepackage{graphicx} % Required for inserting images
\usepackage{tabularx} % For figures and tables

\usepackage[english]{babel} %set the language to English
\usepackage{lipsum}
\usepackage{caption}
\usepackage{float}
\usepackage{xstring}
\usepackage{pdfpages}
\usepackage{wrapfig}
\usepackage{hyperref}
\usepackage{longtable}
\usepackage{csquotes}


\usepackage{glossaries}

\graphicspath{ {./assets/} }

% for page numbering style
\renewcommand\thesection{\arabic{section}}

% Make glossaries
\makenoidxglossaries
\usepackage{pages/dict}

% Makes the text of glossary entries italic
\renewcommand{\glstextformat}[1]{\textit{#1}}

\setcounter{secnumdepth}{4}

%include this so that your paragraphs don't indent automatically
\setlength\parindent{0pt}

% No spacing between list items
\setlist{noitemsep}

\linespread{1.2}% line distance

% --------------- OPTIONAL ---------------------
\numberwithin{equation}{section} % Number equations within sections (i.e. 1.1, 1.2, 2.1, 2.2 instead of 1, 2, 3, 4)
\numberwithin{figure}{section} % Number figures within sections (i.e. 1.1, 1.2, 2.1, 2.2 instead of 1, 2, 3, 4)
\numberwithin{table}{section} % Number tables within sections (i.e. 1.1, 1.2, 2.1, 2.2 instead of 1, 2, 3, 4)
% ----------------------------------------------

%----------------------------------------------------------------------------------------
%	DOCUMENT MARGINS
%----------------------------------------------------------------------------------------

\usepackage{geometry} % Required for adjusting page dimensions and margins

\geometry{
	paper=a4paper, % Paper size, change to letterpaper for US letter size
	top=2.5cm, % Top margin
	bottom=3cm, % Bottom margin
	left=3cm, % Left margin
	right=3cm, % Right margin
	headheight=0.75cm, % Header height
	footskip=1.5cm, % Space from the bottom margin to the baseline of the footer
	headsep=0.75cm, % Space from the top margin to the baseline of the header
	%showframe, % Uncomment to show how the type block is set on the page
}

%----------------------------------------------------------------------------------------
%	FONTS
%----------------------------------------------------------------------------------------

\usepackage[utf8]{inputenc} % Required for inputting international characters
\usepackage[T1]{fontenc} % Use 8-bit encoding

\usepackage{fourier} % Use the Adobe Utopia font for the document

%----------------------------------------------------------------------------------------
%	SECTION TITLES
%----------------------------------------------------------------------------------------

\usepackage{sectsty} % Allows customising section commands

\sectionfont{\vspace{6pt}\centering\normalfont\scshape} % \section{} styling
\subsectionfont{\normalfont\bfseries} % \subsection{} styling
\subsubsectionfont{\normalfont\itshape} % \subsubsection{} styling
\paragraphfont{\normalfont\scshape} % \paragraph{} styling

%----------------------------------------------------------------------------------------
%	HEADERS AND FOOTERS
%----------------------------------------------------------------------------------------

\usepackage{scrlayer-scrpage} % Required for customising headers and footers

\ohead*{} % Right header
\ihead*{} % Left header
\chead*{} % Centre header

\ofoot*{} % Right footer
\ifoot*{} % Left footer
\cfoot*{\pagemark} % Centre footer

%----------------------------------------------------------------------------------------
%	CITING STUFF
%----------------------------------------------------------------------------------------
%use this exact command. The style and bibliographystyle has to be author year (Harvard). The sorting is nyt: name, year, title so that the bibliography is sorted alphabetically. firstInits=true shortens the names: Albert Einstein -> A. Einstein
\usepackage[backend=bibtex,style=authoryear,bibstyle=authoryear,sorting=nyt,firstinits=true,uniquename=init]{biblatex}
% Declare custom cite command for personal communication (optional)
\DeclareAutoCiteCommand{inline}{\mycite}{\cites}
\DeclareCiteCommand{\mycite}
 {}{\ifentrytype{misc}{%
 \IfStrEq{\thefield{howpublished}}{personal communication}
 {\mkbibparens{\printnames{labelname}, personal communication, \printdate}}
 {\mkbibparens{\usebibmacro{cite}}}%
 }
 {\mkbibparens{\usebibmacro{cite}}}%
 }{}{}
% End of custom cite command % Include the file specifying the document structure and custom commands
\addbibresource{main.bib}

%----------------------------------------------------------------------------------------
%	TITLE SECTION
%----------------------------------------------------------------------------------------

\title{	
	\normalfont\normalsize
	\textsc{University}\\ % Your university, school and/or department name(s)
	\vspace{25pt} % Whitespace
	\rule{\linewidth}{0.5pt}\\ % Thin top horizontal rule
	\vspace{20pt} % Whitespace
	{\huge Some title}\\ % The assignment title
	\vspace{12pt} % Whitespace
	\rule{\linewidth}{2pt}\\ % Thick bottom horizontal rule
	\vspace{12pt} % Whitespace
}

\author{\LARGE Author Name} % Your name

\date{\normalsize\today} % Today's date (\today) or a custom date

\begin{document}

\maketitle % Print the title
\newpage

%----------------------------------------------------------------------------------------
%	ToC and so on
%----------------------------------------------------------------------------------------

\pagenumbering{gobble}

\setcounter{tocdepth}{3}
\tableofcontents
\newpage

\listoffigures
\newpage

\listoftables
\newpage

\printnoidxglossaries
\cleardoublepage
\pagenumbering{arabic}
\setcounter{section}{1}

%------------------------------------------------

\section{Writing sample}

\subsection{Example paragraph}
While this question leaves out the crucial element of the geographic origin of the swallow, according to Jonathan Corum, an unladen European swallow maintains a cruising airspeed velocity of \textbf{11 metres per second}, or \textbf{24 miles an hour}. The velocity of the corresponding African swallows requires further research as kinematic data is severely lacking for these species.~\parencite{some_article}

%----------------------------------------------------------------------------------------
%	EQUATION EXAMPLES
%----------------------------------------------------------------------------------------

\section{Interpreting Equations}

\subsection{Identify the author of Equation \ref{eq:bayes} below and briefly describe it in English.}

\begin{align} 
	\label{eq:bayes}
	\begin{split}
		P(A|B) = \frac{P(B|A)P(A)}{P(B)}
	\end{split}					
\end{align}

Lorem ipsum dolor sit amet, consectetur adipiscing elit. Praesent porttitor arcu luctus, imperdiet urna iaculis, mattis eros. Pellentesque iaculis odio vel nisl ullamcorper, nec faucibus ipsum molestie. Sed dictum nisl non aliquet porttitor. Etiam vulputate arcu dignissim, finibus sem et, viverra nisl. Aenean luctus congue massa, ut laoreet metus ornare in. Nunc fermentum nisi imperdiet lectus tincidunt vestibulum at ac elit. Nulla mattis nisl eu malesuada suscipit.

%------------------------------------------------

\subsection{Try to make sense of some more equations.}

\begin{align} 
	\begin{split}
		(x+y)^3 &= (x+y)^2(x+y)\\
		&=(x^2+2xy+y^2)(x+y)\\
		&=(x^3+2x^2y+xy^2) + (x^2y+2xy^2+y^3)\\
		&=x^3+3x^2y+3xy^2+y^3
	\end{split}					
\end{align}

Lorem ipsum dolor sit amet, consectetuer adipiscing elit. 
\begin{align}
	A = 
	\begin{bmatrix}
		A_{11} & A_{21} \\
		A_{21} & A_{22}
	\end{bmatrix}
\end{align}
Aenean commodo ligula eget dolor. Aenean massa. Cum sociis natoque penatibus et magnis dis parturient montes, nascetur ridiculus mus. Donec quam felis, ultricies nec, pellentesque eu, pretium quis, sem.

%----------------------------------------------------------------------------------------
%	LIST EXAMPLES
%----------------------------------------------------------------------------------------

\section{Viewing Lists}

\subsection{Bullet Point List}

\begin{itemize}
	\item First item in a list 
		\begin{itemize}
		\item First item in a list 
			\begin{itemize}
			\item First item in a list 
			\item Second item in a list 
			\end{itemize}
		\item Second item in a list 
		\end{itemize}
	\item Second item in a list 
\end{itemize}

%------------------------------------------------

\subsection{Numbered List}

\begin{enumerate}
	\item First item in a list 
	\item Second item in a list
	\begin{enumerate}
        \item First of second item in a sub-list
        \item Second of second item in a sub-list
    \end{enumerate} 
	\item Third item in a list
\end{enumerate}

%----------------------------------------------------------------------------------------
%	TABLE EXAMPLE
%----------------------------------------------------------------------------------------

\section{Interpreting a Table}

\begin{table}[h] % [h] forces the table to be output where it is defined in the code (it suppresses floating)
	\centering % Centre the table
	\begin{tabular}{l l l}
		\toprule
		\textit{Per 50g} & \textbf{Pork} & \textbf{Soy} \\
		\midrule
		Energy & 760kJ & 538kJ\\
		Protein & 7.0g & 9.3g\\
		Carbohydrate & 0.0g & 4.9g\\
		Fat & 16.8g & 9.1g\\
		Sodium & 0.4g & 0.4g\\
		Fibre & 0.0g & 1.4g\\
		\bottomrule
	\end{tabular}
	\caption{Sausage nutrition.}
\end{table}

%------------------------------------------------

This is how a table works. 

%----------------------------------------------------------------------------------------
%	CODE LISTING EXAMPLE
%----------------------------------------------------------------------------------------

\section{Reading a Code Listing}

\lstinputlisting[
	caption=C++ script for sensor reading., % Caption above the listing
	label=lst:arduino_code, % Label for referencing this listing
	language=C++, % Use language for functions/syntax highlighting
	frame=single, % Frame around the code listing
	showstringspaces=false, % Don't put marks in string spaces
	numbers=left, % Line numbers on left
	numberstyle=\tiny, % Line numbers styling
	]{./code/argus.ino}

%------------------------------------------------

\newpage
\printbibliography[title=References]
\end{document}


